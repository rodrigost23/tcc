% Apresentacao
% ------------------------------------------------------------------------
% ------------------------------------------------------------------------

\title{Monografia TCC}

\documentclass[
	% -- opções da classe memoir --
	12pt,				% tamanho da fonte
	%openright,			% capítulos começam em pág ímpar (insere página vazia caso preciso)
	%twoside,			% para impressão em verso e anverso. Oposto a oneside
    oneside,
	a4paper,			% tamanho do papel.
	% -- opções da classe abntex2 --
	chapter=TITLE,		% títulos de capítulos convertidos em letras maiúsculas
	%section=TITLE,		% títulos de seções convertidos em letras maiúsculas
	%subsection=TITLE,	% títulos de subseções convertidos em letras maiúsculas
	%subsubsection=TITLE,% títulos de subsubseções convertidos em letras maiúsculas
	% -- opções do pacote babel --
	english,			% idioma adicional para hifenização
	% french,				% idioma adicional para hifenização
	% spanish,			% idioma adicional para hifenização
	brazil				% o último idioma é o principal do documento
	]{abntex2}

    %\renewcommand{\ABNTEXpartfontsize}{\normalsize}
	%\renewcommand{\ABNTEXchapterfontsize}{ \large}
	%\renewcommand{\ABNTEXsectionfontsize}{\normalsize}
	%\renewcommand{\ABNTEXsubsectionfontsize}{\normalsize}

% ---
% Pacotes básicos
% ---
\usepackage{lmodern}			% Usa a fonte Latin Modern
\usepackage[T1]{fontenc}		% Selecao de codigos de fonte.
\usepackage[utf8]{inputenc}		% Codificacao do documento (conversão automática dos acentos)
\usepackage{lastpage}			% Usado pela Ficha catalográfica
\usepackage{indentfirst}		% Indenta o primeiro parágrafo de cada seção.
\usepackage{color}				% Controle das cores
\usepackage{graphicx}			% Inclusão de gráficos
\usepackage{microtype} 			% para melhorias de justificação
\usepackage{todonotes}


\definecolor{algColor}{RGB}{255,206,206} % rgb(255, 206, 206)

% Pacotes para algoritmos/pseudo-código
\usepackage{listings}

\renewcommand{\lstlistingname}{Programa}

\lstset{ %Formatting for code in appendix
    language=C,
    basicstyle=\footnotesize,
    numbers=left,
    stepnumber=1,
    frame = single,
    showstringspaces=false,
    tabsize=2,
    breaklines=true,
    xleftmargin=10pt,
    breakatwhitespace=false,
    extendedchars=true,
    literate={á}{{\'a}}1
             {ã}{{\~a}}1
             {é}{{\'e}}1
             {í}{{\'\i}}1
             {õ}{{\~o}}1
             {ç}{{\c{c}}}1,
}

\renewcommand{\lstlistlistingname}{Lista de códigos}

% Configura a ``Lista de Códigos'' conforme as regras da ABNT (para abnTeX2)
\begingroup\makeatletter
\let\newcounter\@gobble\let\setcounter\@gobbletwo{}
  \globaldefs\@ne{} \let\c@loldepth\@ne{}
  \newlistof{listings}{lol}{\lstlistlistingname}
  \newlistentry{lstlisting}{lol}{0}
\endgroup

\renewcommand{\cftlstlistingaftersnum}{\hfill--\hfill}

\let\oldlstlistoflistings\lstlistoflistings{}
\renewcommand{\lstlistoflistings}{%
   \begingroup%
   \let\oldnumberline\numberline%
   \renewcommand{\numberline}{\lstlistingname\space\oldnumberline}%
   \oldlstlistoflistings%
   \endgroup}
% ---
% Pacotes adicionais, usados apenas no âmbito do Modelo Canônico do abnteX2
% ---
\usepackage{lipsum}				% para geração de dummy text
% ---


% ---
% Pacotes de citações
% ---
%\usepackage[brazilian,hyperpageref]{backref}	 % Paginas com as citações na bibl
\usepackage[alf, abnt-etal-cite=2]{abntex2cite}	% Citações padrão ABNT

% ---
% CONFIGURAÇÕES DE PACOTES
% ---

\usepackage{abntex_dcc_ufrr}


% PACOTES DE ALGORITMO

\usepackage{algpseudocode,algorithm}
% Declaracoes em Português

\algrenewcommand\algorithmicend{\textbf{FIM}}
\algrenewcommand\algorithmicdo{\textbf{FAÇA}}
\algrenewcommand\algorithmicwhile{\textbf{ENQUANTO}}
\algrenewcommand\algorithmicfor{\textbf{PARA}}
\algrenewcommand\algorithmicforall{\textbf{PARA TODO}}
\algrenewcommand\algorithmicif{\textbf{SE}}
\algrenewcommand\algorithmicthen{\textbf{ENTÃO}}
\algrenewcommand\algorithmicelse{\textbf{SENÃO}}
\algrenewcommand\algorithmicreturn{\textbf{RETORNE}}
\algrenewcommand\algorithmicfunction{\textbf{FUNÇÃO}}
% New definitions
\algnewcommand\algorithmicswitch{\textbf{ESCOLHA}}
\algnewcommand\algorithmiccase{\textbf{CASO}}
\algnewcommand\algorithmicassert{\texttt{assert}}
\algnewcommand\Assert[1]{\State{} \algorithmicassert(#1)}%
% New "environments"
\algdef{SE}[SWITCH]{Switch}{EndSwitch}[1]{\algorithmicswitch\ #1\ \algorithmicdo}{\algorithmicend\ \algorithmicswitch}%
\algdef{SE}[CASE]{Case}{EndCase}[1]{\algorithmiccase\ #1}{\algorithmicend\ \algorithmiccase}%
\algtext*{EndSwitch}%
\algtext*{EndCase}%


% Rearranja os finais de cada estrutura
\algrenewtext{EndWhile}{\algorithmicend\ \algorithmicwhile}
\algrenewtext{EndFor}{\algorithmicend\ \algorithmicfor}
\algrenewtext{EndIf}{\algorithmicend\ \algorithmicif}
\algrenewtext{EndFunction}{\algorithmicend\ \algorithmicfunction}

% O comando For, a seguir, retorna 'para #1 -- #2 até #3 faça'
\algnewcommand\algorithmicto{\textbf{até}}
\algrenewtext{For}[3]%
{\algorithmicfor\ #1 $\gets$ #2 \algorithmicto\ #3 \algorithmicdo}


% ---
% Configurações do pacote backref
% Usado sem a opção hyperpageref de backref
%\renewcommand{\backrefpagesname}{Citado na(s) página(s):~}
% Texto padrão antes do número das páginas
%\renewcommand{\backref}{}
% Define os textos da citação
%\renewcommand*{\backrefalt}[4]{
%	\ifcase #1 %
%		Nenhuma citação no texto.%
%	\or
%		Citado na página #2.%
%	\else
%		Citado #1 vezes nas páginas #2.%
%	\fi}%
% ---

% ---
% Informações de dados para CAPA e FOLHA DE ROSTO
% ---
\titulo{MONOGRAFIA}

\autor{RODRIGO DOS SANTOS TAVARES}
\local{Boa Vista --- RR}
\data{2017}
\orientador{DSc. Herbert Oliveira Rocha}

\tipotrabalho{Monografia}

\preambulo{Monografia de Graduação apresentada ao Departamento de Ciência da Computação da Universidade Federal de Roraima como requisito parcial para a obtenção do grau de Bacharel em Ciência da Computação.}

% ---

%-- Informações de dado para a FOLHA DE APROVAÇÃO
\renewcommand{\dataDefesa}{?? de ????? de 2017}%TODO:Data de defesa
\renewcommand{\orientadorBanca}{Prof.\ Dr.\ Herbert Oliveira Rocha}
\renewcommand{\primeiroMembroBanca}{Prof.\ ???????}%TODO:Membros da banca
\renewcommand{\segundoMembroBanca}{Prof.\ ???????}

% ---
% Configurações de aparência do PDF final

% informações do PDF
\makeatletter
\hypersetup{
     	%pagebackref=true,
		pdftitle={\@title},
		pdfauthor={\@author},
    	pdfsubject={\imprimirpreambulo},
	    pdfcreator={LaTeX with abnTeX2},
		pdfkeywords={abnt}{latex}{abntex}{abntex2}{trabalho acadêmico},
		colorlinks=true,       		% false: boxed links; true: colored links
    	linkcolor=blue,          	% color of internal links
    	citecolor=blue,        		% color of links to bibliography
    	filecolor=magenta,      	% color of file links
		urlcolor=blue,
		bookmarksdepth=4
}
\makeatother
% ---

% ---
% Espaçamentos entre linhas e parágrafos
% ---

% O tamanho do parágrafo é dado por:
\setlength{\parindent}{1.3cm}

% Controle do espaçamento entre um parágrafo e outro:
\setlength{\parskip}{0.2cm}  % tente também \onelineskip

% ---
% compila o indice
% ---
\makeindex
% ---

% ----
% Início do documento
% ----
\begin{document}

% Retira espaço extra obsoleto entre as frases.
\frenchspacing

% ----------------------------------------------------------
% ELEMENTOS PRÉ-TEXTUAIS
% ----------------------------------------------------------
% \pretextual

% ---
% Capa
% ---
\imprimircapa{}
% ---

% ---
% Folha de rosto
% (o * indica que haverá a ficha bibliográfica)
% ---
\imprimirfolhaderosto{}
% ---

% ---
% Inserir folha de aprovação
% ---
\imprimirfolhadeaprovacao{}
% ---
% Dedicatória
% ---
\begin{dedicatoria}
   \vspace*{\fill}
   \centering
   \noindent
   \textit{???????} \vspace*{\fill}%TODO: Dedicatória
\end{dedicatoria}
% ---

% ---
% Agradecimentos
% ---
\begin{agradecimentos}
  \lipsum[1]%TODO:Agradecimentos
\end{agradecimentos}
% ---

% ---
% Epígrafe
% ---
\begin{epigrafe}
    \vspace*{\fill}
	\begin{flushright}
		\textit{???????????} %TODO: Epígrafe
	\end{flushright}
\end{epigrafe}
% ---

% ---
% RESUMOS
% ---

% resumo em português
\setlength{\absparsep}{18pt} % ajusta o espaçamento dos parágrafos do resumo
\begin{resumo}
  \lipsum[1] %TODO: Resumo

 \vspace{\onelineskip}

 \noindent
 \textbf{Palavras-chaves}: ??????%TODO:Palavras-chave
\end{resumo}

\begin{resumo}[Abstract]
 \begin{otherlanguage*}{english}
   \lipsum[1] %TODO: Abstract

   \vspace{\onelineskip}

   \noindent
   \textbf{Key-words}: ???????%TODO:Keywords
 \end{otherlanguage*}
\end{resumo}
% ---
% inserir lista de ilustrações
% ---
\pdfbookmark[0]{\listfigurename}{lof}
\listoffigures*
\cleardoublepage{}
% ---

% ---
% inserir lista de tabelas
% ---
\pdfbookmark[0]{\listtablename}{lot}
\listoftables*
\cleardoublepage{}
% ---

% ---
% inserir lista de abreviaturas e siglas
% ---
\begin{siglas}
  \item[????] ??????%TODO:Lista de siglas
%    \item[EDS] Exemplo De Sigla
\end{siglas}
% ---

% ---
% inserir lista de símbolos
% ---
% \begin{simbolos}
%   \item[$ \Gamma $] \todo{Atualizar esta lista!} Letra grega Gama
%   \item[$ \Lambda $] Lambda
%   \item[$ \zeta $] Letra grega minúscula zeta
%   \item[$ \in $] Pertence
% \end{simbolos}
% % ---

% ---
% inserir o sumario
% ---
\pdfbookmark[0]{\contentsname}{toc}
\tableofcontents*
\cleardoublepage{}
% ---



% ----------------------------------------------------------
% ELEMENTOS TEXTUAIS
% ----------------------------------------------------------
\textual{}
% ----------------------------------------------------------
% Introdução
% ----------------------------------------------------------
\chapter{INTRODUÇÃO}
%% ----------------------------------------------------------
% INTRODUÇÃO
% ----------------------------------------------------------
Soluções tecnológicas são desenvolvidas cada vez mais para suprir necessidades que melhorem a qualidade de vida das pessoas. Quando nos tornamos incapazes de interagir fisicamente com o ambiente ao nosso redor, buscamos este tipo de solução. O campo da robótica de reabilitação trabalha sobre esta ideia, visando trazer conforto e restaurar a independência de pessoas com diversos tipos de limitações, incluindo pessoas com membros amputados, que precisam de próteses para voltar às atividades cotidianas~\cite{siciliano:2008}.

Segundo \citeonline{siciliano:2008}, o desafio do desenvolvimento de próteses de membros humanos é manter a funcionalidade de um membro natural. Aspectos desta naturalidade incluem controle intuitivo das articulações, controle da força dos membros para situações diversas, e sentidos tátil e de movimento que existem nos membros naturais.

As próteses mais comuns para membros inferiores são otimizadas para caminhada em linha reta e, por isso, têm rigidez fixa nas articulações, o que dificulta várias ações do cotidiano que fogem desse comportamento padrão \cite{pew:2017}. Conforme a análise de \citeonline{dedic:2011}, próteses comerciais ainda costumam ser passivas mesmo com o avanço tecnológico dos últimos anos, e muitas funções motoras exigem uma energia maior nas articulações.

De acordo com \citeonline{novak:2013Automated}, vários dispositivos que visam aprimorar ou restaurar funções motoras de membros inferiores foram desenvolvidos, incluindo exoesqueletos e próteses ativas. Estes sistemas são equipados com sensores utilizados para perceber o ambiente ao redor do corpo humano. Além de sensores localizados nos dispositivos em si, também existem sistemas com sensores como acelerômetros\todo{Adicionar referência}\ montados no corpo do usuário, com o intuito de compreender as intenções do usuário e prever seus movimentos.

Para que ações como subir escadas e rampas sejam realizadas naturalmente, é necessário que se utilize mais energia nas articulações, que precisam de mais força nessas ações do que na caminhada plana. Mesmo com o controle computadorizado de articulações que permitem velocidades diferentes de caminhada, ainda é difícil para pessoas amputadas a subida de escadas \cite{dedic:2011}.

Com o advento das próteses ativas, tornou-se possível definir estratégias de controle diferentes para cada tipo de ação do usuário. Estas estratégias são necessárias porque a biomecânica da perna é bem variável em relação à ação realizada, dependendo se o indivíduo está caminhando em um chão plano, ou subindo ou descendo uma rampa ou escada, por exemplo. Um campo de pesquisa atual consiste em determinar precisa e rapidamente o tipo de ação que o usuário está realizando, sem que seja necessário um dispositivo externo à prótese \cite{stolyarov:2017}.

Ainda segundo \citeonline{stolyarov:2017}, os métodos mais eficazes de prever estas ações de caminhada atualmente envolvem o reconhecimento de padrões de alguns sensores como unidades de medição inerciais (IMU, em inglês), e eletrodos de eletromiografia superficial (sEMG), que é um método que varia muito fora de ambientes de laboratório, pois depende de diversos fatores fisiológicos.

%Motivação
Dado este contexto, é pertinente o desenvolvimento de novas técnicas em robótica de reabilitação que procurem melhorar a qualidade de vida de pessoas com certas necessidades. Este trabalho visa contribuir para o desenvolvimento de próteses robóticas que auxiliem em uma interação natural com o ambiente através de um sistema computacional embarcado de baixo custo. Este sistema funcionará por meio de sensores de movimento posicionados nos membros inferiores do usuário, cujos dados serão processados por algoritmos de aprendizado de máquina, para que sejam previstas as ações do indivíduo e seja adaptada a prótese de acordo com a situação.

% ----------------------------------------------------------
% PROBLEMA
% ----------------------------------------------------------
\section{Definição do Problema}

Por conta da dificuldade que pessoas amputadas têm em realizar certas atividades com as próteses mais comuns do mercado, novas formas de auxiliá-las são necessárias para promover seu bem estar. Desta forma, o problema abordado por este trabalho é representado pelo seguinte questionamento: \textbf{Como projetar uma prótese robótica para o pé humano que funcione de forma autônoma através da previsão de movimentos das articulações com o intuito de auxiliar na locomoção de seu usuário, visando um baixo custo e permitindo o monitoramento da saúde de seu usuário em relação a situações ortopédicas de risco?}	

% ----------------------------------------------------------
% OBJETIVOS
% ----------------------------------------------------------
\section{Objetivos}
\label{sec:objetivos}

O objetivo geral deste trabalho é projetar e avaliar uma prótese robótica que seja autônoma por meio da previsão de movimentos das articulações do usuário durante as suas ações executadas, utilizando-se de modelos formais para validar o fluxo de execução do sistema proposto e a análise de dados coletados durante sua execução, bem como a utilização de sensores e atuadores disponíveis no mercado nacional e estruturas feitas em impressora 3D, visando o baixo custo de produção.
% sensores de movimento, pressão, motores disponíveis no mercado nacional e estruturas feitas em impressora 3D, visando o baixo custo de produção.

%O objetivo geral deste trabalho é projetar e desenvolver uma prótese robótica que seja autônoma para usuário por meio da previsão de movimentos de suas articulações durante as ações executadas, ao mesmo tempo deve ser de baixo custo e efetuar monitoramento da saúde de seu usuário.

Os objetivos específicos são os seguintes:
\begin{enumerate}
  \item Identificar métodos para a modelagem do software e do hardware;
  \item Definir um modelo formal que represente o fluxo de execução do sistema proposto, visando analisar propriedades de segurança do funcionamento do sistema;
  \item Demonstrar uma técnica para transformação de modelos de software em códigos do projeto;
  \item Propor um método para prever movimentos de articulações através da classificação de sinais extraídos de membros residuais;
  \item Determinar componentes eletrônicos de baixo custo para a prototipação do sistema proposto;
  \item Propor uma estrutura para prótese utilizando impressora 3D;
  \item Propor uma técnica que analise os dados dos sensores e o funcionamento do sistema, afim de prever perigos à saúde do usuário relacionados ao uso da prótese, por exemplo, postura incorreta;
  \item Validar e avaliar o sistema proposto pela análise de testes práticos e simulados, a fim de examinar a sua eficácia e aplicabilidade.
\end{enumerate}

% ----------------------------------------------------------
% METODOLOGIA
% ----------------------------------------------------------
%\section{Metodologia Proposta}
%\label{sec:metodologia}

% ----------------------------------------------------------
% CONTRIBUIÇÕES
% ----------------------------------------------------------
% \section{Contribuições Propostas}
% \label{sec:contribuicoes}

% ----------------------------------------------------------
% ORGANIZAÇÃO
% ----------------------------------------------------------
\section{Organização do Trabalho}
\label{sec:organizacao}


% ----------------------------------------------------------
% Conceitos e Definições
% ----------------------------------------------------------
\chapter{CONCEITOS E DEFINIÇÕES}
%\input{sections/conceitos_e_definicoes.tex}

% ----------------------------------------------------------
% Revisão de Literatura
% ----------------------------------------------------------
\chapter{TRABALHOS CORRELATOS}
%\input{sections/trabalhos_correlatos.tex}


% ----------------------------------------------------------
% Detalhes de Desenvolvimento do Projeto
% ----------------------------------------------------------
\chapter{MÉTODO PROPOSTO}
%\input{sections/metodologia.tex}

% ----------------------------------------------------------
% Cronograma
% ----------------------------------------------------------
\chapter{CRONOGRAMA}
%\todo[inline]{Fazer cronograma}
\begin{table}[htbp]
  \centering
  \caption{Cronograma de atividades}
  \label{tab:cronograma}
  \begin{tabular}{|c|c|c|c|c|c|}
    \hline
    \textbf{Atividade} & \textbf{Março} & \textbf{Abril} & \textbf{Maio} & \textbf{Junho} & \textbf{Julho} \\
    \hline
    Elaboração do projeto & \(\times\) & & & & \\
    \hline
    Revisão da literatura & & \(\times\) & & & \\
    \hline
    Metodologia & & & \(\times\) & & \\
    \hline
    Conceitos e Definições & & & & \(\times\) & \\
    \hline
    Entrega do trabalho para banca & & & & & \(\times\) \\
    \hline
    Defesa para a banca & & & & & \(\times\) \\
    \hline
  \end{tabular}
  %\legend{Fonte: Autor.}
\end{table}

% ----------------------------------------------------------
% Resultados -- Pode vir junto com discussão
% ----------------------------------------------------------
\chapter{RESULTADOS EXPERIMENTAIS}
%\input{sections/resultados_experimentais.tex}
%\chapter{DISCUSSÃO}

% ----------------------------------------------------------
% Conclusão
% ----------------------------------------------------------
\chapter{CONSIDERAÇÕES PARCIAIS E TRABALHOS FUTUROS}
%\input{sections/consideracoes_parciais.tex}
%\newpage
% ----------------------------------------------------------
% Referências bibliográficas
% ----------------------------------------------------------
\bibliography{main}

%---------------------------------------------------------------------
% INDICE REMISSIVO
%---------------------------------------------------------------------
%\phantompart
\printindex
%---------------------------------------------------------------------

\end{document}
