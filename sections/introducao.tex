% ----------------------------------------------------------
% INTRODUÇÃO
% ----------------------------------------------------------
\todo[inline,color=lightgray]{Contextualizar robótica, falar problemas das próteses (a maioria é passiva, mas existe algumas autônomas -- mas são caras). Apresentar próteses robóticas como solução etc. :)
\par
- Impressoras 3D
\par
\textbf{Incluir Motivação na Introdução!}}



% ----------------------------------------------------------
% PROBLEMA
% ----------------------------------------------------------
\section{Definição do Problema}
\todo[inline]{Escrever a definição do problema acima}%TODO
O problema abordado por este trabalho é representado pelo seguinte questionamento: \textit{Como projetar uma prótese robótica para a perna humana que funcione de forma autônoma através da previsão de movimentos das articulações com o intuito de auxiliar na locomoção de seu usuário, visando um baixo custo e permitindo o monitoramento da saúde de seu usuário em situações de perigo?}	

% ----------------------------------------------------------
% OBJETIVOS
% ----------------------------------------------------------
\section{Objetivos}
\label{sec:objetivos}
%\todo{Revisar, foi atualizado, pois está muito parecido com a questão de pesquisa, aqui temos que uma resposta}
O objetivo geral deste trabalho é projetar e desenvolver uma prótese robótica que seja autônoma por meio da previsão de movimentos das articulações do usuário durante as ações executadas, utilizando-se de modelos formais para validar a ações do sistema proposto e a definição de dados a serem coletados durante sua execução, bem como a utilização de sensores de movimento, pressão, motores disponíveis no mercado nacional e estruturas feitas em impressora 3D, visando o baixo custo de produção.

%O objetivo geral deste trabalho é projetar e desenvolver uma prótese robótica que seja autônoma para usuário por meio da previsão de movimentos de suas articulações durante as ações executadas, ao mesmo tempo deve ser de baixo custo e efetuar monitoramento da saúde de seu usuário.

Os objetivos específicos são os seguintes:
\begin{enumerate}
  \item Identificar métodos para a modelagem do software e do hardware;
  \item Definir um modelo formal que represente o fluxo de execução do sistema proposto, visando analisar propriedades de segurança;
  \item Demonstrar uma técnica para transformação de modelos de software em códigos do projeto;
  \item Propor um método para prever movimentos de articulações através da classificação de sinais extraídos do membro residual;
  \item Determinar componentes eletrônicos de baixo custo para a prototipação do sistema proposto;
  \item Construir uma estrutura para prótese utilizando impressora 3D.
  \item Propor uma técnica que analise os dados dos sensores e o funcionamento do sistema, afim de prever perigos à saúde do usuário relacionados ao uso da prótese como, por exemplo, postura incorreta;
  \item Validar sistema proposto pela análise de testes práticos e simulados, a fim de examinar a sua eficácia e aplicabilidade.
\end{enumerate}

% ----------------------------------------------------------
% METODOLOGIA
% ----------------------------------------------------------
\section{Metodologia Proposta}
\label{sec:metodologia}

% ----------------------------------------------------------
% CONTRIBUIÇÕES
% ----------------------------------------------------------
\section{Contribuições Propostas}
\label{sec:contribuicoes}

% ----------------------------------------------------------
% ORGANIZAÇÃO
% ----------------------------------------------------------
\section{Organização do Trabalho}
\label{sec:organizacao}
