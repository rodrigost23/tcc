\section{Sistemas Embarcados}
Sistemas computacionais fazem parte de diversos produtos desde computadores pessoais e \textit{laptops}, até utensílios domésticos. Destes sistemas, são conhecidos como embarcados os que fazem parte de um dispositivo eletrônico maior, não representando sua totalidade, mas oferecendo recursos computacionais. \cite{vahid:2002}

Entende-se por sistema embarcado um sistema computacional de propósito específico -- em contraste a um computador pessoal --, que não assume vários papéis de acordo com a necessidade do usuário, apenas realiza tarefas predeterminadas. \cite{heath:2002}


\subsection{Microcontroladores vs. Microprocessadores}
\subsection{IoT: Internet das Coisas}

\section{Robótica e suas Aplicações}
\subsection{Próteses Robóticas}
\todo[inline]{Próteses ativas, passivas, eletromiográficas}
\section{Eletromiografia: Geração de Dados}
\todo[inline]{Como funciona, de onde vêm os dados, etc.}

\section{Reconhecimento de Padrões}
\subsection{Pré-processamento}
\todo[inline]{Filtros e fusão de dados, provavelmente?}%TODO
\subsection{Aprendizado de máquina}
\todo[inline]{Dizer aqui que nas próximas seções serão abordados alguns dos principais métodos de classificação}%TODO
\subsubsection{Árvores de decisão}
\todo[inline]{SciKit-Learn usa uma versão otimizada do algoritmo CART}%TODO
\subsubsection{SVM e outros}
\todo[inline]{Mas por que não SVM, por exemplo?}
\todo[inline]{É necessário falar sobre impressora 3D, caso eu pretenda usar uma?}
\todo[inline]{Herbert: Não falaremos na introdução}

\section{Modelos Formais para Validação de Sistemas}
\subsection{Automatos}
\subsection{Redes de Petri}

\section{Modelagem de software}
\subsection{Transformações de modelo para código}