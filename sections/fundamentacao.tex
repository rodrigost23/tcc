\section{Sistemas Embarcados}
\label{sec:embarcados}
Sistemas computacionais fazem parte de diversos produtos desde computadores pessoais e \textit{laptops}, até utensílios domésticos. Destes sistemas, são conhecidos como embarcados os que fazem parte de um dispositivo eletrônico maior, não representando sua totalidade, mas oferecendo recursos computacionais \cite{vahid:2002}. Entende-se por sistema embarcado um sistema computacional de propósito específico -- em contraste a um computador pessoal --, que não assume vários papéis de acordo com a necessidade do usuário, apenas realiza tarefas predeterminadas \cite{heath:2002}. Isto é, são sistemas eletrônicos projetados para funções específicas dentro de um dispositivo maior, que pode controlar o meio físico e permite sua interação com o usuário.

\todo{Fazer gancho para próxima seção}
\subsection{Microcontroladores vs. Microprocessadores}
Microcontroladores são processadores com vários componentes embutidos, como RAM, uma memória de programa e interfaces de entrada e saída \cite{white:2011}. Os microcontroladores surgiram como um substituto para circuitos lógicos discretos, por serem mais facilmente programáveis e proporcionarem uma funcionalidade maior \cite{heath:2002} e, segundo \citeonline{marwedel:2010}, grande parte dos processadores em sistemas embarcados são, na verdade, microcontroladores, por serem simples e fáceis de usar.

A distinção entre microprocessadores e microcontroladores é difícil de ser feita, \citeonline{schlett:1998} diz que, de forma simplificada, é comum diferenciá-los considerando seu desempenho, considerando dispositivos de 8 e 16-bit como microcontroladores, e a partir disto como microprocessadores.

\todo{Falta alguma coisa, talvez.}	

\subsection{IoT: Internet das Coisas}
Com a constante evolução e a ubiquidade da Internet, começam a surgir objetos conectados, transformando dispositivos que já faziam parte do cotidiano em algo que possa ser autônomo e inteligente \cite{kopetz:2011}.  Segundo \citeonline{xia:2012}, o termo ``Internet das Coisas''  não refere-se apenas à existência destes dispositivos inteligentes, mas à interconexão dos objetos do cotidiano através de sistemas embarcados, proporcionando assim ambientes em que os dispositivos se comunicam entre si e também com seres humanos.

\section{Robótica e suas Aplicações}
\label{sec:robotica}
\subsection{Próteses Robóticas}
\todo[inline]{Próteses ativas, passivas, eletromiográficas}

\section{Eletromiografia: Geração de Dados}
\label{sec:emg}
\todo[inline]{Como funciona, de onde vêm os dados, etc.}

\section{Reconhecimento de Padrões}
\label{patternrec}
Nesta seção serão abordados aspectos relevantes ao reconhecimento dos padrões dos sinais, desde seu pré-processamento a previsões dos movimentos do usuário.
\subsection{Pré-processamento}
\todo[inline]{Filtros e fusão de dados, provavelmente?}%TODO
\subsection{Aprendizado de máquina}
\todo[inline]{Dizer aqui que nas próximas seções serão abordados alguns dos principais métodos de classificação}%TODO
\subsubsection{Árvores de decisão}
\todo[inline]{SciKit-Learn usa uma versão otimizada do algoritmo CART}%TODO
\subsubsection{SVM e outros}

\todo[inline]{É necessário falar sobre impressora 3D, caso eu pretenda usar uma?}
\todo[inline]{Herbert: Não, falaremos na introdução}

\section{Modelos Formais para Validação de Sistemas}
\label{sec:modelosformais}
\subsection{Autômatos}
\subsection{Redes de Petri}

\section{Modelagem de software}
\label{sec:modelagem}
\subsection{Transformações de modelo para código}