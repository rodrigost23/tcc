\section{Sistemas Embarcados}
\label{sec:embarcados}
Sistemas computacionais fazem parte de diversos produtos desde computadores pessoais e \textit{laptops}, até utensílios domésticos. São conhecidos como sistemas embarcados (SE) os sistemas computacionais que fazem parte de um dispositivo eletrônico maior, não representando sua totalidade, mas oferecendo recursos computacionais específicos para o seu funcionamento \cite{vahid:2002}. Também pode-se definir um SE como um sistema computacional de propósito específico -- em contraste a um computador pessoal --, que não assume vários papéis de acordo com a necessidade do usuário, apenas realiza tarefas predeterminadas \cite{heath:2002}. Isto é, são sistemas eletrônicos projetados para funções específicas dentro de um dispositivo maior, que pode controlar o meio físico e permite sua interação com o usuário.

\textcolor{red}{ADICIONAR TEXTO SOBRE EXEMPLOS DE SE.}

\textcolor{red}{ADICIONAR TEXTO SOBRE RESTRIÇÕES DE SE.}

\textcolor{red}{ADICIONAR TEXTO SOBRE O USO DE SE NA ATUALIDADE.}

\todo{Fazer gancho para próxima seção}
\subsection{Microcontroladores vs. Microprocessadores}
Microcontroladores são processadores com vários componentes embutidos, como RAM, uma memória de programa e interfaces de entrada e saída \cite{white:2011}. Os microcontroladores surgiram como um substituto para circuitos lógicos discretos, por serem mais facilmente programáveis e proporcionarem uma funcionalidade maior \cite{heath:2002} e, segundo \citeonline{marwedel:2010}, grande parte dos processadores em sistemas embarcados são, na verdade, microcontroladores, por serem simples e fáceis de usar.

A distinção entre microprocessadores e microcontroladores, contudo, não é tão trivial: \citeonline{schlett:1998} diz que, de forma simplificada, é comum diferenciá-los tendo como parâmetro seu desempenho, considerando dispositivos de 8 e 16-bit como microcontroladores, e a partir disto como microprocessadores.

\todo{Falta alguma coisa, talvez.}	

\subsection{IoT: Internet das Coisas}
Com a constante evolução e a ubiquidade da Internet, começam a surgir objetos conectados, transformando dispositivos que já faziam parte do cotidiano em algo que possa ser autônomo e inteligente \cite{kopetz:2011}.  Segundo \citeonline{xia:2012}, o termo ``Internet das Coisas''  não refere-se apenas à existência destes dispositivos inteligentes, mas à interconexão dos objetos do cotidiano através de sistemas embarcados, proporcionando assim ambientes em que os dispositivos se comunicam entre si e também com seres humanos.


\section{Robótica e suas Aplicações}
\label{sec:robotica}
A robótica é uma área que busca sintetizar atividades humanas através do uso de mecanismos, sensores, atuadores e computadores. É comum que pesquisas neste campo seja feita por pesquisadores de outras áreas, servindo como meio para diversos fins \cite{craig:2005}. Sistemas embarcados são tradicionalmente usados na área da robótica, relacionando-se aos aspectos mecânicos \cite{marwedel:2010}. \citeonline{craig:2005} divide a robótica em quatro grandes áreas: manipulação mecânica, locomoção, visão computacional e inteligência artificial.
\todo[inline]{Uma aplicação da robótica seria na construção de próteses etc.}

\subsection{Próteses Robóticas}
\todo[inline]{Faltando aqui ainda}


\section{Eletromiografia: Geração de Dados}
\label{sec:emg}
A eletromiografia (EMG) é uma técnica que consiste em monitorar atividade neuromuscular, sendo possível assim detectar os potenciais de ação através desta leitura \cite{deluca:1979}. Em teoria, isto significa que, a partir do eletromiograma, é possível identificar a intenção de uma pessoa ao realizar uma atividade motora.

A leitura dos sinais eletromiográficos pode ser feita de diferentes formas: por profundidade ou por superfície. O primeiro método consiste em inserir uma agulha que atinja o músculo desejado, permitindo a captação dos sinais; já a eletromiografia por superfície (sEMG) utiliza apenas um conjunto eletrodos sobre a pele, na região do músculo desejado.\todo{Falta referências aqui!}

\todo[inline]{''sEMG é útil pra ler dados de membro residual de pessoas amputadas``}

O eletromiograma consiste em [...]\todo{Explicar a saída dos sinais, o que são os números etc.}


\section{Reconhecimento de Padrões}
\label{sec:patternrec}
O reconhecimento de padrões é um campo que busca encontrar padrões em conjuntos de dados, com finalidade em diversas áreas, e vem sendo desenvolvido ao lado de Aprendizado de Máquina no decorrer dos anos. Com isto, a intenção é usar algoritmos computacionais no intuito de descobrir automaticamente regularidades em dados, possibilitando, por exemplo, a classificação de tais dados \cite{bishop:2006}.

Os dados eletromiográficos podem ser usados para classificação a partir do reconhecimento de padrões. Esta seção abordará o aprendizado de máquina e os algoritmos envolvidos na classificação dos movimentos a partir dos dados dos sensores.
\subsection{Pré-processamento}
\todo[inline]{Filtros ou fusão de dados... Depende do sensor a ser utilizado!}%TODO
\subsection{Aprendizado de máquina}
\todo[inline]{Dizer aqui que nas próximas seções serão abordados alguns dos principais métodos de classificação}%TODO
\subsubsection{Árvores de decisão}
\todo[inline]{SciKit-Learn usa uma versão otimizada do algoritmo CART}%TODO
\subsubsection{SVM e outros}

\todo[inline]{É necessário falar sobre impressora 3D, caso eu pretenda usar uma?}
\todo[inline]{Herbert: Não, falaremos na introdução}


\section{Modelos Formais para Validação de Sistemas}
\label{sec:modelosformais}
\subsection{Autômatos}
\subsection{Redes de Petri}


\section{Modelagem de software}
\label{sec:modelagem}
\subsection{Transformações de modelo para código}