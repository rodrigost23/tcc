O presente trabalho abordou o desenvolvimento de um simulador de prótese robótica para o pé humano baseado em reconhecimento de movimentos através do uso de sensores em um sistema embarcado e utilização de técnicas de aprendizado de máquina, com o intuito de facilitar o desenvolvimento de próteses ativas. Desta maneira, foi criado um protótipo a fim de realizar experimentos e avaliar a eficácia do método proposto.

A partir dos experimentos realizados, constatou-se que o método apresentado é eficaz no que se propõe: prever ações do usuário com o uso de sensores, e exibir uma simulação contando com uma prótese virtual cujos atuadores são acionados em tempo real de acordo com a classificação dos movimentos. Dentro dos cenários avaliados, foi necessária a captura dos dados individuais dos usuários para garantir que o sistema fosse capaz de identificar cada ação.

Como propostas de trabalhos futuros, propõe-se a produção de um conjunto de dados maior para propiciar um modelo de previsão de movimentos genérico para mais usuários e \ldots\todo{Não sei mais}{} Adicionalmente, deverão ser investigadas formas de analisar a saúde do usuário de acordo com diretrizes ortopédicas, para que sejam usadas usadas em um relatório gerado a partir dos dados dos sensores.