Este trabalho abordou o desenvolvimento de uma prótese robótica para o pé humano com o intuito de melhorar a qualidade de vida de pessoas com amputações, através do uso de aprendizado de máquina.

No decorrer do trabalho, foi possível observar várias abordagens a problemas semelhantes, que servirão de base para tomar certas decisões como o algoritmo de aprendizado de máquina a ser utilizado.

Foi planejado o funcionamento da prótese, incluindo o fluxo de funcionamento do \textit{software}, contendo a classificação e previsão de movimentos do usuário. Neste contexto, foi elaborado um diagrama de sequência.

No futuro, planeja-se elaborar uma rede de Petri a partir do diagrama de sequência, para que possa ser validada antes que o \textit{software} seja desenvolvido. Em seguida, será construído o \textit{software} e confeccionada a prótese e todo o sistema de sensores e atuadores. Assim, será possível fazer o treinamento do modelo de aprendizado de máquina.

Também deverão ser investigadas formas de analisar a saúde do usuário de acordo com diretrizes ortopédicas, para que sejam usadas usadas em um relatório gerado a partir dos dados dos sensores.