Este capítulo apresentará outros trabalhos existentes na literatura que compartilham de objetivos semelhantes ou são comparáveis de alguma forma e contribuem para o desenvolvimento deste trabalho. Os trabalhos a seguir propõem métodos para previsão de movimentos para próteses ativas.

\section{Translational Motion Tracking of Leg Joints for Enhanced Prediction of Walking Tasks}
\cite{stolyarov:2017}

\section{Turn Intent Detection for Control of a Lower Limb Prosthesis}
\cite{pew:2017}

\section{Automated detection of gait initiation and termination using wearable sensors}
\cite{novak:2013Automated}

\section{A Novel Design of a Full Length Prosthetic Robotic Arm for the Disabled}
\cite{kumar:2017}

\section{SmartLeg: An intelligent active robotic prosthesis for lower-limb amputees}
Foi proposto por \citeonline{dedic:2011} uma forma de transformar uma prótese passiva disponível comercialmente em ativa, para que seja possível a subida e descida de escadas e outros movimentos que exigem energia externa. Como para subir escadas é importante a presença do joelho e de mais força, faz-se necessário o uso de uma fonte de energia externa, que foi feita usando atuadores hidráulicos nas articulações. \citeonline{dedic:2011} também discutem o uso de aprendizagem de máquina para garantir conforto aos usuários, pois pode-se adaptar os atuadores para funcionarem melhor com os padrões de andadura da pessoa, mas isso não foi implementado.

\section{Evaluation of shoulder complex motion-based input strategies for endpoint prosthetic-limb control using dual-task paradigm}
\cite{losier:2011}

\section{Movement error rate for evaluation of machine learning methods for sEMG-based hand movement classification}
\cite{gijsberts:2014}
